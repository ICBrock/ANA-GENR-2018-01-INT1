\section{Analysis Phases}
\label{sec:analysis-phases}

For many years ATLAS has been using Glance to support the approval process of Papers, CONF and PUB notes. Since then they were considered independent systems, so they had different creation and search interfaces and also different workflows, all starting from its own Phase 1. As recently many projects at CERN changed their version control software from SVN to Gitlab, this has triggered the need of new Glance/Fence interface that could automatically create and configure Gitlab repositories. Thus, this new interface was created and called Phase 0.

The Phase 0 is common to Papers, CONF and PUB notes workflows, before Phase 1. It stores some metadata divided in steps, for example, meeting dates, comments and links, the groups of people there are Analysis Contact and their target date for Analysis finalisation, the groups of people there are Editorial board, approval sign-off dates and others. It also stores some basic metadata as short and full title, leading group, subgroups, Analysis Team members and supporting documents that will be after propagated to the publication (Paper, CONF note or PUB note) that proceeds Phase 0.

The Phase 0 system is where all the Gitlab environment for an ATLAS publication is set. When Phase 0 starts, a Gitab group is created to store one or more repositories related to the publication that should be written. A repository for one first internal document is also created with some template files and variables correctly substituted. The access control of those groups and repository are also set by the system, being the group of the Analysis Team defined in Phase 0. Also, some branches of the repositories are correctly created and protected.

In addition to the first internal document repository, more repositories can be created through Phase 0 system. One Phase 0 can have one or more internal document and one for a CONF note, PUB note or Paper. Detectors groups, those who need PUB notes or even Internal note for local reviews in sub-systems don’t have the necessity to go through the Phase 0 workflow, so they can to skip it and go directly to the Glance creation of a any type of publication, that, normally, happens at the end of Phase 0 workflow.

It can happen that a Phase 0 never evolves to become a Paper, CONF or PUB note, and turn into an internal document. In this case, the workflow will stop in Phase 0 and only internal documents repositories will exists. Also, if at any point the group convener decides that the Analysis should be discontinued, the deletion of the Phase 0 can be requested.

In some cases there is the need to create a Phase 0 entry very similar with one that already exists. The system allows that the existent entries can be cloned into new Phase 0 entries.