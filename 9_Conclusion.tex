% !TEX root = ANA-GENR-2018-01-INT1.tex
% Turn off some chktex warnings.
% chktex-file 1 chktex-file 8 chktex-file 46

%------------------------------------------------------------------------------
\section{Conclusion}
\label{sec:Conclusion}
%------------------------------------------------------------------------------

TO BE REWRITTEN

In view of the amount of publications ATLAS generates, it is clear that software tools helps to better manage the writing, approval and submission process of those documents.
The new Analysis Phase 0 system made possible to centralise in one interface a process before it is spread in many emails exchanged, formalising it a default workflow.
The \gitlab integration avoided the need of each writer creates their own repository susceptible to configuration errors, having standardised infrastructure based on templates.
The \gitlab CI tools included in all publications repositories assure that the documents produced satisfies the journal rules, favouring the submission process before done manually by the Physics Office. 
The proofs checker is also making the Physics Office work easier,
as it helps to discover the errors in the author lists and acknowledgements, avoiding the officers having to check them one by one.
All those developments result a more efficient work to the involved, and this was this project's main goal.

