\section{Introduction}
\label{sec:Introduction}
The Physics and Committees Office (PO) is an instance of ATLAS collaboration constituted of physicists and engineers performing several tasks in view of a continuous support to ATLAS committees and groups such as ATLAS management, Physics coordinators, Publication Committee chairs, Analysis Group conveners, Authorship committee, Speakers 
committee and many others.
                    
The PO also provides support to any member of the ATLAS collaboration. It is helping with membership, authorship, paper submission to arXiv and journals, reviewing talks and posters to National and Regional meetings, designing or participating to the development of many tools such as the Analysis and Papers handling systems, the Performance Plots and Proceedings Trackers, Gitlab for document’s edition, author lists and acknowledgements creation and maintenance. PO assists many other daily tasks to lower the load on each member of the Collaboration, including the Physics Office itself.

ATLAS collaboration~\cite{atlas_collab}, comprises around 5000 active people from which there are around 3000 scientific authors. Consequently, a dedicated organisation of the work, detector maintenance and operation, data analysis and scientific publication and outreach has been set up, as described in section~\ref{sec:ATLAS_work_strategy}. Collaborative tools are thus needed to perform not only an efficient communication between the collaborators but also an easy manageable interaction with the outside world, namely the public notes, the publication journals, the institutions, the funding agencies, etc.
                    
In this report, we will focus on the Analysis and Papers infrastructure and its most recent developments launched in fall 2017, described in section~\ref{sec:The_FENCE_project} and \ref{sec:Analysis_and_paper_phases}. Indeed, due to the phase out of the SVN system~\cite{svn} , the developers built a user friendly tool to handle any analysis and document type (for internal use or for large publication) using the FENCE framework  together with GitLab~\cite{gitlab}, as described at section~\ref{sec:PO-Gitlab_and_CI_tools}. The FENCE Gitlab integration is described at section~\ref{sec:FENCE_and_Gitlab_integration}. In section~\ref{sec:Authorlists_Acknowledgements_and_ProofChecker}, a description of the main tools used to handle the collaboration author list and the funding and foundation agencies acknowledgements is done. A more general description on the way we manage the metadata is presented in section~\ref{sec:Handling_the_metadata}.