% !TEX root = ANA-GENR-2018-01-INT1.tex
% Turn off some chktex warnings.
% chktex-file 1 chktex-file 8 chktex-file 46

\section{Introduction}%
\label{sec:Introduction}

The ATLAS Physics and Committees Office (PO) is constituted of physicists and engineers performing several tasks connected to the continuous support of committees and groups such as ATLAS management, Physics Coordinators, Publication Committee chairs, Analysis Group conveners, Authorship Committee, Speakers
Committee and many others. The ATLAS collaboration~\cite{PERF-2007-01}, comprises around 5000 active people from which there are around 3000 scientific authors.

The PO also provides help to any member of the ATLAS collaboration. These include helping with membership, authorship, paper submission to arXiv and journals, reviewing talks and posters to national and regional meetings, designing or participating to the development of many tools such as those used to handle the management of the Analysis, preparing and submitting the Papers, the Performance Plots and Proceedings Trackers, using version control (Git) for editing documents, handling authorship and  author lists and the last but not the least, being alert  and available to assist the users when necessary, in case of bugs in the system for example. PO assists also on many other daily tasks to lower the load on each member of the collaboration, including the Physics Office itself.

A dedicated organisation of the work, detector maintenance and operation, data analysis and scientific publication and outreach has been set up, as described in \cref{sec:ATLAS_work_strategy}.
Collaborative tools are thus needed to perform not only an efficient communication between the collaborators but also an easy manageable interaction with the outside world, namely the publication journals, the institutions, the funding agencies.

In this report, we focus on the Analysis and Papers infrastructure and its most recent developments launched in fall 2017, described in \cref{sec:The_FENCE_project} and \cref{sec:Analysis_and_paper_phases}.
Due to the phasing out of the SVN system~\cite{svn}, we built a new system to handle any analysis and document type (for internal use or for a large publication) using the FENCE framework  together with GitLab~\cite{gitlab}, as described in \cref{sec:PO-Gitlab_and_CI_tools}.
The FENCE--GitLab integration is described in \cref{sec:FENCE_and_Gitlab_integration}. In \cref{sec:Authorlists_Acknowledgements_and_ProofChecker}, a description of the main tools used to handle the collaboration author list and the funding and foundation agencies acknowledgements is given.
A more general description of the way we manage the metadata is presented in \cref{sec:Handling_the_metadata}.
