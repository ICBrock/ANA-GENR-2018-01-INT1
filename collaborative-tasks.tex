\section{ATLAS Publication and collaborative tasks}
\label{sec:collaborative-tasks}
Fence is a PHP framework designed after is was noticed that many of the Glance systems had similar features and the developers team turnover was big. Thus, the goal of the Fence project was to create a Web framework in which the Glance management systems could be developed in a way to favour the development of software requirements, reducing maintenance costs and speeding up the development process.

Fence was built using oriented object programming what made possible the creation of many classes that could be used among the Glance systems. One example is the GlanceSearch class allowing the creation of search interfaces with only some lines of code and the specification of the search attributes in a configuration file. The SuperSearch class provides an advanced search interface, where the user can build logic queries with AND and OR operators. The User class supports the access control of the interfaces. The Mailer class can be used in order to send emails from the interfaces. Form inputs can be easily added using the classes TextInput, to add text inputs, DateInput, to add date inputs, MemberInput, to add a selection box with a list of all the members of one experiment, and many others already implemented.

The Fence Workflow class is one more of the in the set that can be inherited by the systems using the framework. That can be used to program any process involving proceeding states and actions triggered while moving to a state to another. This is largely used by the ATLAS Analysis systems, that are divided in phases and those are divided in many steps. Each of those steps can record metadata in the ATLAS database, trigger an egroup creation/update, trigger a Gitlab repository creation/update/modification or send automatic emails. 