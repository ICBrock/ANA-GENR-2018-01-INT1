% !TEX root = ANA-GENR-2018-01-INT1.tex
% Turn off some chktex warnings.
% chktex-file 1 chktex-file 8 chktex-file 46

%------------------------------------------------------------------------------
\section{ATLAS Publication Process strategy}%
\label{sec:ATLAS_work_strategy}
%------------------------------------------------------------------------------

%------------------------------------------------------------------------------
%\subsection{ATLAS analyses and publications}%
%\label{sec:The_ATLAS_Analyses_and_Publications}

ATLAS experiment supports a  general physics programme to explain the nature of matter. To do so,  it makes use of the gigantic hadron accelerator (LHC), which collides protons at almost the speed of light. With the centre-of-mass energy of \SI{13}{\TeV} provided by the accelerator, we explore matter, its interaction, its properties and simulate small big bangs every second. To perform such a physics program, physicists need software and graphics tools to analyse the data and compare them to the different models on the market.

For this, ATLAS is organised in several Physics (PHY) and  Combined Performance (CP) working groups and subgroups. These groups are coordinated by conveners, who are elected or appointed by the collaboration for one or more years.
For example, some  PHY and CP groups are labeled top quark (TOPQ), Standard Model (STDM), \PB physics (BPHY), Higgs (HIGG), Electron/Gamma (EGAM), Jet and EtMiss (JETM).
Further studies on system detectors (SYS) or other activities like software (SOFT) and data preparation (DAPR) are also organised hierarchically with subgroups and conveners.

Once an analysis is finished or an aspect of the detector performances has been studied in detail,
the analysis team is asked to prepare a publication.
In fundamental research, as is the case with the research conducted at CERN, the publication of the results is  compulsory and is the only way to show the results publicly.

ATLAS considers three different types of publications:
\begin{itemize}
    \item[$\bullet$] general publications based on data and to project-related  (PAPER);
    \item[$\bullet$] public documents classified as notes (PUB notes);
    \item[$\bullet$] conference proceedings (PROC) or notes (CONF Notes) on preliminary results, which are shown at conferences.
\end{itemize}

All ATLAS analyses are discussed and presented in the relevant working groups  which have the responsibility, together with the subgroups, to provide guidance, help and/or resources to the analyses during the Analysis early stages and during their development. The working groups should also develop a coherent and realistic plan for the release of the results for a conference and/or for a journal publication. This is a necessary step before any paper draft can be planned or circulated. This process' phases are described in details in the \textit{ATLAS procedures for the approval of physics results} document ~\cite{Pub-policy}, but, in summary:

\begin{itemize}

    \item[$\bullet$] The launch of an analysis or a document is done at the so-called \textbf{Phase 0}. The Analysis Team starts the analysis writing and, consequently, they should request repositories to edit any type of document (PAPER, CONF, PUB) and as many INT (internal) drafts. In parallel, some important definitions are made such as the constitution of Analyses groups, the appointments of group and sub group conveners, analysis contacts and the Editorial Board (EdBoard).
    
    \item[$\bullet$] The EdBoard reviews the complete analysis and ensures that both the supporting documentation and the paper draft are well prepared. Then, it is required to sign off on the supporting documentation (analysis) and draft PAPER or CONF note before its circulation to ATLAS. The EdBoard should verify that the analysis is worth publishing in the proposed form, and consult with the Publications Committee (PubComm) chair if there are doubts. It should also establish with the editors and conveners whether the paper should be a letter or an article, and propose a journal. These tasks and validation steps are done during the called \textbf{Phase 1} and \textbf{Phase 2}, related respectively to the 1st and 2nd circulation to the collaboration, two sequences of the process of a validation of a publication. In the circulation period, all signing authors of the collaboration are expected to read and comment on paper and/or note drafts.
    
    \item[$\bullet$] The PubComm chair has the responsibility to assess the quality of the paper and to ensure ATLAS guidelines and policies are followed. After the Public Reading, and the sign-off by the EdBoard, the draft goes to the chair of PubComm for a final sign-off.
    
    \item[$\bullet$] The Spokesperson (SP) is ultimately responsible for the scientific quality of the results from the ATLAS collaboration and has a final look at each paper before the \textbf{Submission}. The Final draft  is signed-off by the SP or its delegate.
    
    \item[$\bullet$] As soon as the SP has signed-off, the validation workflow at Phase 2 is finished and sends a message to the Physics Office Publications crew (PO-PUB) to announce them an expected submission. PO-PUB officers then proceed with the submission to arXiv and the Journal. They also keep the communication with the journal during all the steps (referree reports and proofs) with the journal through a dedicated workflow which ends the final publication and the publishing of the journal references.
    
\end{itemize}

For the CONF or PUB notes there are no PubComm or SP reviewers. Instead, the Physics Coordinator appoints two members of the collaboration to review the draft and perform the final sign-off. These two types of documents need only Phase 1.

These tasks and validation steps are translated as a workflow in the FENCE systems called Analysis, that encompass all Phase 0, 1, 2 and Submission steps. Those systems are described in \cref{sec:Analysis_and_paper_phases}, but focusing on Phase 0. Using Analysis/Phase 0 system, dedicated to Phase 0 tasks, editors can request the creation of configured repositories due to the FENCE-GitLab integration, as described in \cref{sec:FENCE_and_Gitlab_integration}. All the phases information can be exported to Twiki pages and Web sites to display the so-called Public Results pages but also to a more private and internal sites to allow the monitoring of the journal submission process. The GitLab CI tools, which are explained in \cref{sec:PO-Gitlab_and_CI_tools}, validates if the Analyses are in the correct format to be submitted in journals.

For the category PAPER, another more longer process is performed by the PO-PUB officers. They check the pertinence of the author list and the acknowledgement which are added to the LaTex sources before the submission to arXiv and the journal. Author lists and Acknowledgements are both handled and generated, using the FENCE framework, and their production is described in detail in \cref{sec:Author_lists_and_acknowledgments_files}. Before the final publication, and after the referee review and acceptance by the Chief Editor of the chosen journal, proofs are sent back to the collaboration for a last check.  While the paper's editors proof-read the content of the paper withing the very short allowed time-lap, usually two days, PO-PUB crew check whether the authors and their affiliations are appropriately handled by the journal, compared to the original author list sent to them. This check is performed automatically using a tool named  the Proof Checker which is described in \cref{sec:Proof_checker_functionalities}.