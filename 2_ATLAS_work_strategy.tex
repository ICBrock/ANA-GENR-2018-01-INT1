% !TEX root = ANA-GENR-2018-01-INT1.tex
% Turn off some chktex warnings.
% chktex-file 1 chktex-file 8 chktex-file 46

%------------------------------------------------------------------------------
\section{ATLAS work strategy}%
\label{sec:ATLAS_work_strategy}
%------------------------------------------------------------------------------

%------------------------------------------------------------------------------
%\subsection{ATLAS analyses and publications}%
%\label{sec:The_ATLAS_Analyses_and_Publications}

ATLAS experiment supports a  general physics programme to explain the nature of matter. To do so,  it makes use of the gigantic hadron accelerator (LHC), which collides protons at almost the speed of light. With the centre-of-mass energy of \SI{13}{\TeV} provided by the accelerator, we explore matter, its interaction, its properties and simulate small big bangs every second. To perform such a physics program, physicists need software and graphics toolsto analyse the data and compare them to the different models on the market.

For this, ATLAS is organised in several Physics (PHY) and  Combined Performance (CP) groups and subgroups. These groups are coordinated by conveners, who are elected or appointed by the collaboration for one or more years.
For example, some  PHY and CP groups are labeled top quark (TOPQ), Standard Model (STDM), \PB physics (BPHY), Higgs (HIGG), Electron/Gamma (EGAM), Jet and EtMiss (JETM).
Further studies on system detectors (SYS) or other activities like software (SOFT) and data preparation (DAPR) are also organised hierarchically with subgroups and conveners.

Once an analysis is finished or an aspect of the detector performances has been studied in detail,
the analysis team is asked to prepare a publication.
In fundamental research, as is the case with the research conducted at CERN, the publication of the results is  compulsory and is the only way to show the results publicly.

ATLAS considers three different types of publications:
\begin{itemize}
    \item[$\bullet$] general publications based on data and to project-related  (PAPER);
    \item[$\bullet$] public documents classified as notes (PUB notes);
    \item[$\bullet$] conference proceedings (PROC) or notes (CONF Notes) on preliminary results, which are shown at conferences.
\end{itemize}

All ATLAS analyses are discussed and presented in the relevant working groups  (physics, combined performance, systems and detectors, etc.). The Physics Coordinator, the working groups conveners and the Publication Committee members are informed of all ongoing analyses and publications. They also appoint analysis contacts, contact editors, review experts and the Editorial Board which is in charge to launch the publication process as described in this  document~\cite{Pub-policy} .

The working groups and subgroups have the responsibility to provide guidance, help and/or resources to the analyses during their early stages and throughout their development. A review of the analysis by the working group takes place throughout the development phase. The working groups should also develop a coherent and realistic plan for the release  of the results for a conference and/or for a journal publication. This is a necessary step before any paper draft can be planned or circulated. The constitution of Analyses groups, the appointments or group and sub group conveners, analysis contacts and the Editorial Board members are done using the\textbf{ Fence} system which is being is described in \cref{sec:The_FENCE_project}. Analyses and Documents are handled through different phases as described in \cref{sec:Analysis_and_paper_phases}.

To summarize:
\begin{itemize}

    \item[$\bullet$]
The launch of an analysis or a document is done at the so-called \textbf{PHASE~0} of the system. PHASE 0 is a special system which is integrated with Fence. Consequently, editors can already request repositories to edit any type of document (PAPER, CONF, PUB) and as many INT (internal)  drafts. The Fence-Gitlab system is described in \cref{sec:FENCE_and_Gitlab_integration}.

    \item[$\bullet$]
    
The Editorial Board (EdBoard) reviews the complete analysis and ensures that both the supporting documentation and the paper draft are well prepared. The EdBoard is required to sign off on the supporting documentation (analysis) and draft PAPER or CONF note before its circulation to ATLAS. The EdBoard should verify that the analysis is worth publishing in the proposed form, and consult with the PubComm (Publications Committee) chair if there are doubts. It should also establish with the editors and conveners whether the paper should be a letter or an article, and propose a journal. These tasks and validation steps are translated as validation workflow in Fence system in theirs \textbf{PHASE~1} and \textbf{PHASE~2}, related respectively to the 1st and 2nd circulation to the collaboration, two sequences of the process of a validation of a publication.

All signing authors of the ATLAS collaboration are expected to read and comment on paper and/or note drafts.

    \item[$\bullet$]

The Publication Committee (PubComm) chair has the responsibility to assess the quality of the paper and to ensure ATLAS guidelines and policies are followed. After the Public Reading, and the sign-off by the EdBoard of a draft  that addresses all comments made to draft-1 (PHASE~1) and draft-2 (PHASE~2)  and at the Public Reading, the draft goes to the chair of PubComm for a final sign-off.

    \item[$\bullet$]

The Spokesperson is ultimately responsible for the scientific quality of the results from the ATLAS collaboration and has a final look at each paper before the \textbf{SUBMISSION}. The Final draft  is signed-off by the spokesperson (SP) or its delegate. 

    \item[$\bullet$]
    
As soon as the SP has signed-off, the validation workflow at Phase 2 is finished and sends a message to the Physics Office Publications crew (PO-PUB) to announce them an expected submission. PO-PUB officers use the \textbf{SUBMISSION} phase of the system and proceed with the submission to arXiv and the Journal. They also keep the communication with the journal during all the steps (referree reports and proofs) with the journal through a dedicated  workflow which ends  the final publication and the publishing of the journal references. The Fence Submission phase is also there to record all the information at each step of the publication step and spread that information to other Twiki and Web sites to display the so-called Public Results pages but also to a more private and internal sites to allow the monitoring of the Analysis, the Publication and the Submission processes.

\end{itemize}

For the CONF or PUB notes there are no PubComm or Spokesperson reviewers. Instead the Physics Coordinator appoints two members of the collaboration to review the draft and perform the final sign-off. These two types of documents need only PHASE~1.

For the category PAPER,  another more longer process is performed by the PO-PUB officers. They check the pertinence of the author list and the acknowledgement which are add to the LaTex sources before the submission to  arXiv and the journal. Author lists and Acknowledgements are both  handled and generated, using the FENCE framework, and their production is described in detail in \cref{sec:Author_lists_and_acknowledgments_files}. Before the final publication, and after the referee review and acceptance by the Chief Editor of the chosen journal, proofs are sent back to the collaboration for a last check.  While the paper's editors proof-read the content of the paper withing the very short allowed time-lap, usually two days, PO-PUB crew check whether the authors and their affiliations are appropriately handled by the journal, compared to the original author list sent to them. This check is performed automatically using a tool named  the Proof Checker which is described in \cref{sec:Proof_checker_functionalities}.