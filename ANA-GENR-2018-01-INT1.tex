%-------------------------------------------------------------------------------
% This file provides a skeleton ATLAS note.
% \pdfinclusioncopyfonts=1
% This command may be needed in order to get \ell in PDF plots to appear. Found in
% https://tex.stackexchange.com/questions/322010/pdflatex-glyph-undefined-symbols-disappear-from-included-pdf
%-------------------------------------------------------------------------------
% Specify where ATLAS LaTeX style files can be found.
\newcommand*{\ATLASLATEXPATH}{latex/}
% Use this variant if the files are in a central location, e.g. $HOME/texmf.
% \newcommand*{\ATLASLATEXPATH}{}
%-------------------------------------------------------------------------------
\documentclass[NOTE, atlasdraft=true, texlive=2016, UKenglish]{\ATLASLATEXPATH atlasdoc}
% The language of the document must be set: usually UKenglish or USenglish.
% british and american also work!
% Commonly used options:
%  atlasdraft=true|false This document is an ATLAS draft.
%  texlive=YYYY          Specify TeX Live version (2016 is default).
%  coverpage             Create ATLAS draft cover page for collaboration circulation.
%                        See atlas-draft-cover.tex for a list of variables that should be defined.
%  cernpreprint          Create front page for a CERN preprint.
%                        See atlas-preprint-cover.tex for a list of variables that should be defined.
%  NOTE                  The document is an ATLAS note (draft).
%  PAPER                 The document is an ATLAS paper (draft).
%  CONF                  The document is a CONF note (draft).
%  PUB                   The document is a PUB note (draft).
%  BOOK                  The document is of book form, like an LOI or TDR (draft)
%  txfonts=true|false    Use txfonts rather than the default newtx
%  paper=a4|letter       Set paper size to A4 (default) or letter.

%-------------------------------------------------------------------------------
% Extra packages:
\usepackage{\ATLASLATEXPATH atlaspackage}
% Commonly used options:
%  biblatex=true|false   Use biblatex (default) or bibtex for the bibliography.
%  backend=bibtex        Use the bibtex backend rather than biber.
%  subfigure|subfig|subcaption  to use one of these packages for figures in figures.
%  minimal               Minimal set of packages.
%  default               Standard set of packages.
%  full                  Full set of packages.
%-------------------------------------------------------------------------------
% Style file with biblatex options for ATLAS documents.
\usepackage{\ATLASLATEXPATH atlasbiblatex}

% Package for creating list of authors and contributors to the analysis.
\usepackage{\ATLASLATEXPATH atlascontribute}

% Useful macros
\usepackage{\ATLASLATEXPATH atlasphysics}
% See doc/atlas_physics.pdf for a list of the defined symbols.
% Default options are:
%   true:  journal, misc, particle, unit, xref
%   false: BSM, heppparticle, hepprocess, hion, jetetmiss, math, process, other, texmf
% See the package for details on the options.

% Files with references for use with biblatex.
% Note that biber gives an error if it finds empty bib files.
% \addbibresource{ANA-GENR-2018-01-INT1.bib}
\addbibresource{bib/ATLAS.bib}
\addbibresource{bib/CMS.bib}
\addbibresource{bib/ConfNotes.bib}
\addbibresource{bib/PubNotes.bib}

% Paths for figures - do not forget the / at the end of the directory name.
\graphicspath{{logos/}{figures/}}

% Add you own definitions here (file ANA-GENR-2018-01-INT1-defs.sty).
\usepackage{ANA-GENR-2018-01-INT1-defs}

%-------------------------------------------------------------------------------
% Generic document information
%-------------------------------------------------------------------------------

% Title, abstract and document
% Turn off some chktex warnings.
% chktex-file 1 chktex-file 8 chktex-file 46
%-------------------------------------------------------------------------------
% This file contains the title, author and abstract.
% It also contains all relevant document numbers used for an ATLAS note.
%-------------------------------------------------------------------------------

% Title
\AtlasTitle{ATLAS publications and the role of Fence and GitLab}

% Draft version:
% Should be 1.0 for the first circulation, and 2.0 for the second circulation.
% If given, adds draft version on front page, a 'DRAFT' box on top of each other page, 
% and line numbers.
% Comment or remove in final version.
\AtlasVersion{0.1}

% Abstract - % directly after { is important for correct indentation
\AtlasAbstract{%
  The ATLAS Collaboration develops and uses web systems and tools, defines methods, establishes procedures and organizes advisory groups to manage the publication processes of scientific papers, conference papers and public notes.
  The so-called Phase 0 system was implemented using the FENCE framework and is integrated into the CERN Gitlab software repository, to automatically configure workspaces where the analysis can be documented and used by the analysis team and managed by the conveners.
  Continuous integration is used to guide the authors on using the accurate format while writing papers to be submitted to scientific journals.
  The ATLAS Physics and Committees Office provide support to the researchers and facilitate each phase of a publication process, allowing authors to focus on the article contents that describe the results and discoveries of the ATLAS experiment.
}

% Author - this does not work with revtex (add it after \begin{document})
% \author{The ATLAS Collaboration}

% Authors and list of contributors to the analysis
% \AtlasAuthorContributor also adds the name to the author list
% Include package latex/atlascontribute to use this
% Use authblk package if there are multiple authors, which is included by latex/atlascontribute
% \usepackage{authblk}
% Use the following 3 lines to have all institutes on one line
\makeatletter
\renewcommand\AB@affilsepx{, \protect\Affilfont}
\makeatother
% \renewcommand\Authands{, } % avoid ``. and'' for last author
% \renewcommand\Affilfont{\itshape\small} % affiliation formatting
\AtlasAuthorContributor{Fairouz Malek}{c}{Head of Physics Office and coordinator of Phase 0, Glance--\gitlab integration. Participation to the overal design, implementation and management of the project.}
\AtlasAuthorContributor{Ian Brock}{a}{Writer and maintainer of ATLAS \LaTeX\ templates.}
\AtlasAuthorContributor{Tancredi Carli}{b}{Physics Coordinator. Participation to the design, implementation and management of part of the project, namely Phase 0.}
\AtlasAuthorContributor{Nuno Castro}{d}{\pogitlab responsible, development and maintenance.}
\AtlasAuthorContributor{Maurizio Colautti}{f}{Developer: Fence, author list and Glance \gitlab integration, AFS and CDS related matters.}
\AtlasAuthorContributor{Ana Carolina da Silva Menezes}{e}{Developer: Glance and Fence.}
\AtlasAuthorContributor{Gabriel de Oliveira da Fonseca}{e}{Developer: Glance and Fence.}
\AtlasAuthorContributor{Andreas Hoecker}{b}{Deputy Spokesperson in charge to follow PO and Glance Team tasks. Participation to the overal design, implementation and management of the project.}
\AtlasAuthorContributor{Gabriela Lemos Lucidi Pinhao}{e}{Developer/Designer: Glance, Fence, Glance \gitlab integration.}
\AtlasAuthorContributor{Carmen Maidantchik}{e}{Glance Team Supervisor.}
\AtlasAuthorContributor{Gianluca Picco}{f}{Developer: Fence, author list and Glance--\gitlab integration.}
\AtlasAuthorContributor{Marcelo Texeira Dos Santos}{e}{\pogitlab developer and maintenance.}
\affil[a]{Universität Bonn}
\affil[b]{CERN}
\affil[c]{LPSC Grenoble}
\affil[d]{LIP Lisbon}
\affil[e]{Rio de Janeiro}
\affil[f]{Udine}
% ATLAS reference code, to help ATLAS members to locate the paper
\AtlasRefCode{ANA-GENR-2018-01}

% ATLAS note number. Can be an COM, INT, PUB or CONF note
\AtlasNote{ANA-GENR-2018-01-INT1}

% Author and title for the PDF file
\hypersetup{pdftitle={ATLAS document},pdfauthor={The ATLAS Collaboration}}

%-------------------------------------------------------------------------------
% Content
%-------------------------------------------------------------------------------
\begin{document}

\maketitle

\tableofcontents

% List of contributors - print here or after the Bibliography.
%\PrintAtlasContribute{0.30}
%\clearpage

%-------------------------------------------------------------------------------
\section{Introduction}
\label{sec:intro}
%-------------------------------------------------------------------------------

Place your introduction here.  blablabl and bla bla la


%-------------------------------------------------------------------------------
\section{Physics Office Gitlab area}% Juan Pe
\label{sec:pogitlab}
%-------------------------------------------------------------------------------

%-------------------------------------------------------------------------------
\section{The Analysis and Publications Glance interface}%Gabriela
\label{sec:anaglance}
%-------------------------------------------------------------------------------
\begin{itemize}
\item Glance/Fence technology introduction:
http://monografias.poli.ufrj.br/monografias/monopoli10014830.pdf and http://monografias.poli.ufrj.br/monografias/monopoli10021571.pdf

\item Analysis Glance introduction:
For many years ATLAS has been using Glance to support the approval process of Papers, CONF and PUB notes. Since always they were considered independent systems, so they had different creation and search interfaces and also different workflows, all starting from its own Phase 1.

As we recently moved from SVN to Gitlab, this has triggered the need of Glance interface that could automatically create and configure Gitlab repositories.

\item Phase 0 introduction:
Will be common to Papers, CONF and PUB notes workflows, before Phase 1
Will store some metadata, if needed
Will automatically create Gitlab workspace for ANY Paper related document
Will provide egroups so editors can communicate

Now, instead of having 3 different creation interfaces (one for Paper, another for CONF note and another for PUB notes) we have only one where PGC can create a new Analysis that contain a Phase 0.

At Analysis creation, some basic metadata are required as short title, full title, leading group and so on and will be inherited by the publication that the Analysis will turn into in the future.

After the creation of a new Analysis, the first Gitlab integration happens, Glance automatically creates the first internal document repository on Gitlab and editors can already start working on it.

\item Phase 0 workflow introduction:
At the beginning of phase 0 it's possible to add as many internal documents repository as needed, if 1 isn't enough, and also one for a Paper, one for a CONF and one for a PUB.

The Phase 0 itself will be a new workflow with some steps that works very similarly to the ones we have nowadays for Phase 1, 2 or submission, with proceed and save buttons. The last step of this workflow is to create a Glance Paper or CONF note entry.

Detectors groups, those who need PUB notes or even Internal note for local reviews in sub-systems don’t have the necessity to go through the workflow, so they will have the permission to skip the workflow and go directly to the Glance creation of a any type of publication.

Also, if at any point the PGC decides that the Analysis should be discontinued, he/she can requests its deletion or clone into a new Analysis.

\item Building a search interface with Fence: ?
Source in Bruno's thesis

\item Building a workflow using Fence:
Make correlation with Graph theory;
How to define steps and actions: steps contains fields, actions can be save, proceed, skip, rollback, reset, egroups update, automatic messages ...

\end{itemize}
%-------------------------------------------------------------------------------
\section{The Physics Office Glance Gitlab integration}%all
\label{sec:poggintg}
%-------------------------------------------------------------------------------

%-------------------------------------------------------------------------------
\subsection{Auhor lists integration}%Gianluca,Maurizio,Gabriela
\label{sec:auth}
%-------------------------------------------------------------------------------
\begin{itemize}
\item Previous authorlist workflow:
AFS, script and problems.

\item New authorlist workflow:
Illustrating this new workflow, at Paper Phase 1, when an Authorlist is created a commit is done in Paper repository adding the first version of the Authorlist files in all available formats. At Phase 2, if the Authorlist has a new change, like a new exception, a new commit will be done, substituting the previous version of the Authorlist files.
In summary: Gitlab will store the last updated version of the Authorlist files.
Similarly to the interface used nowadays on the Web, Glance has an interface listing all author list already created in all available formats. The difference with the old Web is that some will point to AFS (old archived author lists) and some will point to Gitlab (newly generated author lists through Glance).
Tickets:
https://its.cern.ch/jira/browse/ATGLANCE-1997
https://its.cern.ch/jira/browse/ATGLANCE-1996
https://its.cern.ch/jira/browse/ATGLANCE-1990
https://its.cern.ch/jira/browse/ATGLANCE-1987
https://its.cern.ch/jira/browse/ATGLANCE-1912
https://its.cern.ch/jira/browse/ATGLANCE-1911
https://its.cern.ch/jira/browse/ATGLANCE-1910
https://its.cern.ch/jira/browse/ATGLANCE-1908

Minutes from POGG meeting can help. The email was sent by Fairouz with subject "POGG minutes"
\end{itemize}

You can find some text snippets that can be used in papers in \texttt{template/atlas-snippets.tex}.
Some of the snippets need the \texttt{jetetmiss} option passed to \texttt{atlasphysics}.
%\subsection{\Antikt}

The \antikt algorithm with a radius parameter of $R=0.4$ is used to reconstruct jets with a four-momentum recombination scheme, using \topos as inputs. Jet energy is calibrated to the hadronic scale with the effect of \pileup removed

\subsection{\Topos}

Hadronic jets are reconstructed from calibrated three-dimensional \topos.
Clusters are constructed from calorimeter cells that are grouped together using a topological clustering algorithm.
These objects provide a three-dimensional representation of energy depositions in the calorimeter and implement a nearest-neighbour noise suppression algorithm.
The resulting \topos are classified as either electromagnetic or hadronic based on their shape, depth and energy density.
Energy corrections are then applied to the clusters in order to calibrate them to the appropriate energy scale for their classification.
These corrections are collectively referred to as \textit{local cluster weighting}, or LCW, and jets that are calibrated using this procedure are referred to as LCW jets~\cite{PERF-2012-01}.

\subsection{Grooming}

Trimming removes subjets with $\ptsubji/\ptjet < \fcut$, where \ptsubji is the transverse momentum of the $i^{\text{th}}$ subjet, and $\fcut=0.05$.
Filtering proceeds similarly, but utilises the relative masses of the subjets defined and the original jet. For at least one of the configurations tested, trimming and filtering are both able to approximately eliminate the \pileup dependence of the jet mass.


%-------------------------------------------------------------------------------
\section{Results}
\label{sec:result}
%-------------------------------------------------------------------------------

Place your results here.

% All figures and tables should appear before the summary and conclusion.
% The package placeins provides the macro \FloatBarrier to achieve this.
% \FloatBarrier


%-------------------------------------------------------------------------------
\section{Conclusion}
\label{sec:conclusion}
%-------------------------------------------------------------------------------

Place your conclusion here.


%-------------------------------------------------------------------------------
% If you use biblatex and either biber or bibtex to process the bibliography
% just say \printbibliography here
\printbibliography
% If you want to use the traditional BibTeX you need to use the syntax below.
%\bibliographystyle{bib/bst/atlasBibStyleWithTitle}
%\bibliography{ANA-GENR-2018-01-INT1,bib/ATLAS,bib/CMS,bib/ConfNotes,bib/PubNotes}
%-------------------------------------------------------------------------------

%-------------------------------------------------------------------------------
% Print the list of contributors to the analysis
% The argument gives the fraction of the text width used for the names
%-------------------------------------------------------------------------------
\clearpage
The supporting notes for the analysis should also contain a list of contributors.
This information should usually be included in \texttt{mydocument-metadata.tex}.
The list should be printed either here or before the Table of Contents.
\PrintAtlasContribute{0.30}


%-------------------------------------------------------------------------------
\clearpage
\appendix
\part*{Appendices}
\addcontentsline{toc}{part}{Appendices}
%-------------------------------------------------------------------------------

In an ATLAS note, use the appendices to include all the technical details of your work
that are relevant for the ATLAS Collaboration only (e.g.\ dataset details, software release used).
This information should be printed after the Bibliography.

\end{document}
